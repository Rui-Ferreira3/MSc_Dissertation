%%%%%%%%%%%%%%%%%%%%%%%%%%%%%%%%%%%%%%%%%%%%%%%%%%%%%%%%%%%%%%%%%%%%%%%%
%                                                                      %
%     File: Thesis_Introduction.tex                                    %
%     Tex Master: Thesis.tex                                           %
%                                                                      %
%     Author: Andre C. Marta                                           %
%     Last modified : 13 May 2019                                      %
%                                                                      %
%%%%%%%%%%%%%%%%%%%%%%%%%%%%%%%%%%%%%%%%%%%%%%%%%%%%%%%%%%%%%%%%%%%%%%%%

\chapter{Introduction}
\label{chapter:introduction}

%%%%%%%%%%%%%%%%%%%%%%%%%%%%%%%%%%%%%%%%%%%%%%%%%%%%%%%%%%%%%%%%%%%%%%%%
\section{Motivation}
\label{section:motivation}
FIXME Melhorar

A Real-Time operating system (RTOS) is one of the components that support the execution of applications in a computational system that can improve the performance of that system. RTOS are often designed to meet the timing requirements of real-time systems, therefore, its use is suitable to manage multiple tasks with varying priorities. An RTOS also provides inter-task communication, mutual exclusion, task synchronization, and power modes. These capabilities are crucial for the system's functioning and to manage the system's resources.

Recently, RISC-V has been establishing itself as a standard open-source instruction set architecture (ISA) both in the academic environment as well as in industry implementations \cite{Will_RISC-V_Revolutionize_Computing}. The open-source nature of RISC-V enables a broader and larger community around it that contributes to its development. Its modular and simplistic approach makes it not only easier but also more suitable for systems with resource constraints.

Hardware accelerators are used in embedded systems by adding custom digital circuits to speed-up algorithms that usually are the computational bottleneck. When using hardware accelerators to perform computationally intensive tasks for which they were specifically designed the system is able to better utilize its resources. The use of accelerators can also improve the responsiveness of the system to the time constraints of real-time systems.

In addition to the performance increase that a hardware accelerator provides, the use of an RTOS in such systems will help take advantage of the CPU time that would otherwise be wasted. Using RISC-V based processor will allow the developer to implement a system that is specific for its application, optimizing, even more, the resource utilization of the system.


%%%%%%%%%%%%%%%%%%%%%%%%%%%%%%%%%%%%%%%%%%%%%%%%%%%%%%%%%%%%%%%%%%%%%%%%
\section{Objectives}
\label{section:objectives}
FIXME Mudar os objetivos para refletir melhor a tese

The primary objective of this thesis is to incorporate a RTOS into a RISC-V processor inside an SoC programmed in a field programmable gate array (FPGA), and use the RTOS capabilities to manage and use a hardware accelerator. The initial work is to survey the existing SoC based on RISC-V implementations and RTOS. After choosing a RISC-V core and system on chip (SoC), they will be targeted on an FPGA. The third stage is to implement and evaluate how the RTOS performs in the RISC-V Soc. The following action is to include a hardware accelerator in the system. Finally, the last stage is to evaluate the performance of the system and to observe how the use of an accelerator will impact the performance of the RTOS.


\section{Document Outline}
TODO Document Outline para o projeto final


% mudar forma das frases

%%%%%%%%%%%%%%%%%%%%%%%%%%%%%%%%%%%%%%%%%%%%%%%%%%%%%%%%%%%%%%%%%%%%%%%%


