\chapter{RISC-V Instruction Set Architecture}
\label{chapter:RISC-V}
\section{RISC-V Overview}
% O que é RISC-V?
% Quais são os ISAs básicos?
RISC-V \cite{RISCV_manual_1}, \cite{RISCV_manual_2} is an instruction set architecture (ISA) that is rapidly growing due to the fact that it is an open-source ISA, contrary to other popular ISAs such as ARM's and Intel's x86. The RISC-V ISA is defined to be a base ISA that is restricted to a minimal set of instructions capable of providing a target to compilers, assemblers, operating systems, etc. In addition to the base ISAs, RISC-V International \cite{RISCV_International} also provides standard extensions and the capability to add custom extensions that enable the developer to add additional capabilities for specific applications.

There are four base ISAs that are characterized by the width of the integer registers and by the number of integer registers. The two primary base ISAs are RV32I and RV64I which provide 32-bit and 64-bit address space, respectively. Both have 32 integer registers. The RV32E is a subset of the RV32I that only has 16 registers and was created to support small microcontrollers. The fourth base ISA is RV128I and supports a 128-bit address space. As mentioned before, RISC-V International provides a few extensions that add some functionalities to the base instruction sets. The most significant among them is the 'M' standard extension that adds integer multiplication and division for signed and unsigned integers. Other extensions are the 'A' standard extension that adds atomic instructions, the 'F' standard extension that provides floating-point operations, floating-point status and control registers and floating-point registers and the 'Zicsr' extension that allows access to the control and status register, this enables interrupt and exception handling.

RISC-V also supports privileged levels and extensions for the privileged ISAs. There are currently three privilege levels: User (U), Supervisor (S) and Machine (M). The privilege modes exist to add protection between components of the software stack. The Machine mode has the most privileges and runs implementation specific firmware. The Supervisor mode is typically used by operating systems and it gives access to most of the hardware. The mode that offers the least privileges is the User mode. It's in this mode that user applications generally run. All implementations must provide the Machine mode since this is the only privileged mode that has access to the entire hardware.

%
%\section{RISC-V Standard Extensions}
% Quais são as extensões que existem?
% Quais são as extensões que preciso?
\section{RISC-V Hardware Implementations}
% O que procuro numa implementação?
% tabela com as implementações que existem
There are many hardware implementations of RISC-V, however, for the purposes of this work, the only implementations analyzed were the ones listed on the list \cite{RISCV_list} under the SoC platforms section, since the focus of this thesis is not to develop the SoC itself. Even though there are several implementations listed it's necessary that the chosen SoC is under an open source license, has good documentation, and an active community support.

Three candidate RISC-V implementations were selected to be studied in detail: LiteX \cite{LiteX}, SweRVolf \cite{SweRVolf} and NEORV32 \cite{NEORV32}. Table \ref{tab:comparação RISC} presents a comparison of the main key factors, such as the base ISA, the supported extensions, the provided RAM size, the system clock speed, the supported simulation tool, and the quality of the documentation provided.

\begin{table}[H]
    \centering
    \begin{tabular}{ c | c | c | c }
         & LiteX & SweRVolf & NEORV32\\
        \hline
        Base ISA & RV32I & RV32I & RV32I\\
        RISC-V Extensions & [M][A][F][D][C] & [M][C] & [M][C][Zicsr][Zicntr]\\ 
        \hline
        RAM & 256MB & 128MB & configurable\\
        system clock & 1MHz & 50MHz & 150MHz\\
        \hline
        Simulation tool & Verilator & Verilator & ISIM\\
        \hline
        Support for RTOS & Zephyr & Zephyr, TockOS & Zephyr, FreeRTOS\\
        \hline
        Documentation & decent & good & decent\\
    \end{tabular}
    \label{tab:comparação RISC}
    \caption{Comparison between RISC-V SoC}
\end{table}

Looking at table \ref{tab:comparação RISC} it's possible to see that the NEORV32 SoC is capable of higher clock speeds and has support for two of more relevant RTOS as will be discussed in Chapter \ref{chapter:RTOS}. However, none of these features is enough to make it the obvious choice, since it also is the more complex one. The LiteX SoC has native support for the Zephyr OS, having a guide on how to install and simulate the RTOS on the Zephyr documentation. The SweRVolf SoC has good support and documentation and already supports relevant RTOS, Zephyr and TockOS. SweRVolf SoC is the SoC selected for this thesis work as it offers resources for the integration of the RTOS that reduces the development effort and ensures a correct integration of the chosen RTOS.

\section{Conclusion}
% Qual é implementação que melhor serve para a tese?
% After analyzing and comparing different RISC-V SoC implementations it can be concluded that the best SoC for this thesis is the SeweRVolf SoC. This is due to the fact that it stands out among the others because of its good support and documentation and its compatibility with relevant RTOS. It offers the necessary computation power with its clock frequency of 50MHz and RAM of 128MB. Besides this, the SoC offers resources for the integration of RTOS that reduce the development effort and ensures a stable integration of the chosen OS.


This chapter provided an overview of the RISC-V ISA. It explained the four base ISA, RV32I, RV64I, RV32E and RV128 as well as some of the more relevant standard extensions. It also introduced the three privilege levels of RISC-V, machine, standard and user. They provide different levels of access to the machine, adding protection between components of the software stack.

A survey of the existing RISC-V hardware implementations was conducted, emphasizing SoC platforms. Of the ones listed in \cite{RISCV_list}, the SoC that were studied in more detail were LiteX, SweRVolf and NEORV32. These implementations were selected based on their open-source license, documentation and community support. Finally, the chosen implementation is SweRVolf as it provides sufficient resources for the installation of the RTOS and offers good documentation and installation guides that will greatly aid the work conducted in this thesis.